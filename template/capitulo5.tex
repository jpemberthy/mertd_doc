\chapter{Marco referencial}
El Mercado Público de Valores en Colombia ha pasado por muchas etapas y ha sido manejado por diversas instituciones, las cuales nacen y desaparecen con el correr de los años según cambian las necesidades de quienes intervienen en ellas. Tuvo que pasar un siglo desde la creación de la primera bolsa en el país para que apareciera la Bolsa de Valores de Colombia, institución que hoy representa la unificación del Mercado Bursátil.\cite{RePEc:col:000159:003891}\\

Así como en otros sectores, el desarrollo tecnológico ha jugado un papel fundamental para el desarrollo de la bolsa de valores, pues desde sus inicios quienes han estado sumergidos en este mundo, se han valido de la tecnología de momento para llevar control de las acciones, es así como desde que se presentó una fiebre de especulaciones surgida por las oscilaciones de la tasa de cambio ocurridas a principio de siglo(debido a la guerra de los mil días) y que se desarrollaban todas las noches de 7:00 a 10:00 en el atrio de la catedral de la Candelaria de Medellín y dentro del parque Pedro Justo Berrio\cite{RepEEc:col:piedrahita}, posteriormente culminaban con el desarrollo de negocios en oro y los registros eran calculados con una máquina sumadora o en algunos casos eran parte de un calculo mental. Actualmente la Bolsa cuenta con avanzados sistemas de información de uso interno que permite conocer el estado en tiempo real de una acción y ver como se ha comportado en el tiempo, las personas naturales que poseen acciones y que no tienen acceso a estos sistemas pueden consultar el estado de sus inversiones en diversos sitios públicos (sitios web, revistas, periódicos), en donde se les muestra una información plana a veces acompañada de unos gráficos en dos dimensiones.\\

Muchas de las personas naturales mencionadas anteriormente, llevan el control de sus inversiones en libros y hojas de cálculo diseñadas por ellos mismos, algunos, los mas familiarizados con herramientas tecnológicas utilizan herramientas disponibles (Google Finance entre otras), estas ultimas herramientas son útiles a la hora de llevar el historial y permiten calcular fácilmente ganancias/pérdidas, algunos de los problemas que tienen estas aplicaciones son: 
\begin{enumerate}
\item Los gráficos a pesar de que son claros, están limitados a presentar en dos dimensiones múltiples indicadores, que a la larga le podría representar al usuario común complicaciones a la hora de entender el gráfico.
\item La mayoría no permiten llevar el control personalizado del portafolio, y aquellas que integran esta característica rara vez incluyen a la bolsa de valores de Colombia o son de libre acceso.
\item Ninguna de estas aplicaciones ofrece un componente 3D que permita comparar varias variables que afectan el mercado accionario para apreciar mejor el comportamiento de las acciones, y que aparte de esto se puedan acceder desde un navegador web, pues la tendencia en 3D es desarrollar aplicaciones tipo \textbf{Stand-Alone}\footnote{Aquellos programas que no necesitan de un interpretador para ser ejecutados en una máquina.}
\end{enumerate}

Es por esto que es pertinente el desarrollo de una aplicación que sea portable y permita ser visualizada desde cualquier lugar en cualquier plataforma, ademas de esto que permita apreciar las variables que afectan el mercado accionario Colombiano.