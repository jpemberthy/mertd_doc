\chapter{Introducción}

\begin{quote} \textit{``A good Notation has a sublety and suggestiveness which at
  times makes it almost seem like a live teacher.''}
  \\ --Bertrand Russell\footnote{Prólogo del capítulo \textit{``Minilanguages: Finding a
    Notation that Sings''} en \textit{``The art of Unix Programming''} por Eric Raymond.}
\end{quote} 

La idea de permitir al usuario crear y modificar operadores dentro del lenguaje de
programación tiene como objetivo permitir sintaxis más concisa. Algunos lenguajes de
programación permiten al programador definir sus propios operadores
infijos(~\cite{jones:haskell},~\cite{leroy:ocaml}). Otros permiten el concepto más
general de operador disfijo(~\cite{Coq},~\cite{goguen:obj},~\cite{clavel:maude} y
otros). Operadores disfijos son descritos en~\cite{jones:distfix} y
~\cite{Aasa:UDS}. Escencialmente se trata de operadores de una o mas partes que
pueden traer los argumentos embebidos en si mismos. Un ejemplo tomado de OBJ es el 
siguiente:

\begin{verbatim}
op if_then_else_fi : Bool Int Int -> Int .
\end{verbatim}

La idea de este trabajo se centra en el estudio de los operadores permisivos 
como mecanismo de extensibilidad en un lenguaje de programación. 

El capitulo 1 es un breve recuento de los objetivos generales y específicos. El
capítulo 2 hace un recorrido sobre diferentes conceptos necesarios para este
trabajo. El capítulo 3 justifíca este trabajo en el área de diseño de lenguajes de
programación. El capítulo 4 explica los detalles de diseño y diferentes estrategias
utilizadas para el \textit{``parsing''} y evaluación de operadores distfijos
permisivos. El capítulo 5 ilustra como puede utilizarse el lenguaje de programación
como mecanísmo de extensibilidad para construir características
pseudo-objetuales. Finalmente el capitulo 6 describe las conclusiones respecto la
implementación de dicho lenguaje y describe posible trabajo futuro alrededor de la
inclusión de operadores permisivos.












  


%%% Local Variables: 
%%% mode: latex
%%% TeX-master: "tesis"
%%% End: 
