\chapter{Importancia del problema.}

Muchos de los avances en el desarrollo de software han tenido su origen en el area de
lenguajes de programación. Los desarrollos en lenguajes de programación han
permitido:

\begin{enumerate}
\item Escribir software en forma portable gracias a la posibilidad de abstraer los
  detalles arquitectónicos de la plataforma para la cual se desarrolla el software.
\item Escribir software mas seguro, ocultando y restringiendo ciertas construcciones
  inseguras del lenguaje objeto.\footnote{Entendiendo lenguaje objeto como el lenguaje
    producido por el lenguaje de programación. Por ejemplo, \textbf{x86} es uno de
    los lenguajes objeto de \textbf{C}}
\item Reutilización de código, mediante la incorporación de abstracciones para la
  generalización de soluciones. \footnote{Un ejemplo de ésto es Polimorfismo. El
    lenguaje diseñado en esta tésis tiene características polimórficas. Más sobre
    esto en la sección 4.}
\end{enumerate}

De la misma manera, desarrollos en lenguajes de programación han permitido el
desarrollo de lenguajes mas expresivos. Entendiendo por lenguaje expresivo es aquel
que brinda construcciones sintácticas y abstracciones que permiten al programador
resolver problemas en forma mas ``elegante''
\footnote{Elegante puede tener connotaciones subjetivas, sin embargo el concepto es
  formalizado en \cite{Chaitin:ElegantLisp}}.

La importancia de la elegancia en los lenguajes de programación además de tener
ventajas en mantenibilidad, y legibilidad tambien parece ser pieza importante en la
reducción de errores. Existen indicios~\cite{ulf:4fold} para afirmar que la
densidad de errores por número de líneas no se ve alterada por la elección del
lenguaje de programación. De esta forma, lenguajes que requieren menos líneas de
código para resolver diferentes problemas (lenguajes expresivos) pueden ayudar a 
disminuir la cantidad de errores. 

Brindar expresividad enfrenta al diseñador del lenguaje de programación al dilema de
Cardelli~\cite{cardelli:extensiblesyntax}: Decidir entre dar una amplia y expresiva
notación o tener un \textit{``core''} pequeño. Ambas propiedades son deseables en un
lenguaje de programación. Los \textit{``cores''} pequeños son mas fáciles de mantener
y permiten que el lenguaje sea asimilado más fácil por los programadores. Existe un
enfoque híbrido basado en lenguajes extensibles. Estos lenguajes parten de un
\textit{``core''} pequeño, pero permiten que el usuario defina su propia sintáxis. 

Esfuerzos para construir lenguajes que puedan ``crecer''~\cite{GuySteele:grow} se han
realizado anteriormente. Dentro de las técnicas utilizadas en este enfoque hibrido
cabe mencionar las siguientes: \textit{``Syntax Macros''} ~\cite{Swierstra:Macros},
\textit{``Extensible Syntax''}~\cite{cardelli:extensiblesyntax},
\textit{``Conctypes"} ~\cite{Aasa:UDS}.

Este trabajo complementa dichos esfuerzos mostrando como un subconjunto de operadores
disfijos puede ser utilizado como mecanismo de extensibilidad en los lenguajes de
programación. 




