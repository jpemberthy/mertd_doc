\chapter{Conclusiones}
\begin{itemize}

\item[$\bullet$] Se puede aprovechar las nuevas tecnologías de información, las teorías de ambientes virtuales y las ventajas que ofrece la web 2.0 para potenciar el entendimiento del estado actual de las inversiones. De esta forma se puede observar que las plataformas web pueden ser un gran apoyo para llevar a cabo los procesos de visualización, propiciando entornos que no eran imaginados hace unos años atrás.

\item[$\bullet$] Las gráficas 3D son herramientas de toma de decisiones en el mercado bursátil mas poderosas que las convencionales gráficas bidimensionales, ya que estas ofrecen una visión global del movimiento accionario.

\item[$\bullet$] Las herramientas tecnológicas de visualización + Web2.0 necesitan estar apoyadas en unos lineamientos metodológicos para ayudar a potencializar y agilizar el desarrollo, de otra manera, tener una plataforma que no este testeada o que presente varios problemas, es incurrir en mas esfuerzos y gastos en el proceso de desarrollo

\item[$\bullet$] Aplicar un estándar para el desarrollo de aplicaciones Web 2.0 como el \texttt{TDD} trae grandes ventajas y facilita el manejo y desarrollo de la aplicación, pues de esta forma la misma tendrá unas características mejor desarrolladas en cuanto a reusabilidad, accesibilidad y seguimiento al usuario.

\item[$\bullet$] Antes de desarrollar una aplicación es necesario conocer todo el modelo de negocio de la institución cliente, ya que sin hacer un análisis del entorno es imposible modelar una solución que se adapte a las necesidades específicas y a las características de los usuarios. Del mismo modo, realizar un proceso de planeación previo al desarrollo del producto es importante para distribuir de una forma mejor los recursos disponibles.

\item[$\bullet$] Por medio de la visualización 3D se pueden observar clara y didácticamente los patrones de fluctuación del mercado	

\item[$\bullet$] Utilizar herramientas y plataformas existentes para adaptarlas a las necesidades específicas de la aplicación solicitada trae mucha facilidad, ahorro de tiempo y esfuerzo, pues de esta forma se puede enfocar en las nuevas funcionalidades y en el modelado del problema. En nuestro caso especifico, utilizar y apoyarnos en \emph{Rails} + \emph{Unity3D} para ofrecerle al usuario los servicios requeridos de interacción (Agregar acciones, ganancias diarias, ganancias totales.) fue de gran utilidad porque no fue necesario implementar el sistema desde cero y no se necesitaron crear elementos ya existentes. De esta forma nos pudimos enfocar primordialmente en las necesidades específicas de los jugadores en la bolsa, permitiendo acotar el problema y trayendo finalmente beneficios que se vieron reflejados en el ahorro de tiempo y facilidad de la implementación.	
	
\end{itemize}
