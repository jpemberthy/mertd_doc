\chapter{Definición del problema}

Durante la evolución de la computación moderna se han presenciado avances significativos en los diferentes campos que componen la misma, uno de estos campos es la visualización de aplicaciones, es decir la forma en la que una aplicación es desplegada para que una persona interactue con una máquina con el fin de cumplir una determinada tarea, el surgimiento de la computación gráfica como área de estudio le ha significado a la computación una nueva vía para desarrollar aplicaciones en las que la visualización juega un papel fundamental y que antes apenas si eran soñadas.\\

Con éste proyecto de grado se busca aprovechar la capacidad tecnológica actual para desarrollar contenidos en tercera dimensión que puedan ser visualizados desde cualquier lugar, el tema sobre el que se trabajó fue el movimiento del mercado de acciones Colombiano, se eligió este sector para brindarle a los propietarios de acciones que no están sumergidos de lleno en este mundo una herramienta que les permita identificar fácil y gráficamente el movimiento del mercado, pues actualmente muchas de estas personas llevan control de sus inversiones en un papel, o en un archivo plano, sobre el que construyen una tabla, y/o gráficas limitadas a dos dimensiones.\\
	
Las gráficas en segunda dimensión, son simples y fáciles de entender cuándo se están comparando acciones en una unidad de tiempo ó se esta viendo el movimiento de una sola acción en un período determinado, pero a la hora de comparar varias acciones en un período, una gráfica en 2D podría ser difícil de entender para un usuario común, mientras que una gráfica en 3D facilita el entendimiento de la misma, con el fin de apoyar el problema definido anteriormente, en esta aplicación se diseñaron un conjunto de contenidos en tercera dimensión que recrean la actividad del mercado, dichos contenidos son alimentados desde una base de datos que se actualiza constantemente, tanto datos como contenidos 3D fueron integrados en una aplicación web permitiendo que la aplicación sea visualizada desde cualquier lugar en el que se tenga acceso a Internet.\\