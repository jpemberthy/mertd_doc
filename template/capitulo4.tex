\chapter{Justificación}
El uso de tecnologías de visualización, como apoyo al entendimiento de datos planos en diferentes entornos, tanto de ingeniería como administrativos es un tema que día a día ha venido tomando auge en nuestra sociedad, debido a que facilitan la comprensión de como se están comportando dichos datos para consecuentemente ayudar en un momento determinado con la toma de decisiones.\\

Al facilitar una herramienta que permita visualizar los datos del mercado de acciones, no solo en el tiempo para una sola acción, sino también entre diferentes acciones a la vez, se abre la posibilidad de comparar rendimientos o pérdidas de una o varias acciones a través del tiempo, dentro de un mismo gráfico, para así mirar tendencias y tomar decisiones.\\

El planteamiento de la solución acá expuesta se basa en la necesidad de tener mas control y conocimiento sobre un portafolio que una persona determinada maneje, independientemente de donde esta se encuentre, pues gracias a la ayuda de la tecnología y mas específicamente de la Web 2.0, acceder desde cualquier computador a un mismo portafolio es completamente transparente para quien lo usa. De esta forma, los usuarios podrán tener acceso a su portafolio de acciones en tiempo real, en el que encuentran tablas y gráficos que representan el historial de sus inversiones. 
